\chapter{Model-based Approach to Estimate Gait Characteristics} \label{chapter:IMM}
The target population of this work is individuals with iSCIs, so the gait velocities of interest are very small in magnitude ($ \sim $ 0.5 m/s) \cite{nymark2005electromyographic}. There is limited work that uses physics-based models as a fundamental component in intent detection. The main attraction of using physics-based models is to allow for the accommodation of a greater number of users by modifying a small set of parameters such as mass, leg length, angles, and stiffness. This contribution uses simplified models of locomotion known as template models \cite{full1999templates} as the basis of an estimation framework.

There are often physical changes to the gait as a result of changes in intended velocity, such as changes in step frequency, step length, and CoM trajectories \cite{kuo2001simple}. Template models emulate the salient characteristics of legged locomotion such as CoM trajectories and ground reaction force (GRF) profiles \cite{mochon1980ballistic}. Therefore, they may also emulate the changes in CoM trajectories and step length corresponding to changes in intent. The purpose of this contribution is to estimate gait characteristics such as gait phase and velocity through the comparison of sensor measurements to gaits from models of bipedal locomotion. The main idea behind the estimation framework developed is to estimate a person's desired gait velocity and phase by comparing the measured CoM trajectory to those in a gait library. The challenges were the increased difficulty in finding low-speed gaits during gait library generation and handling the hybrid dynamics of legged locomotion used to describe the different gait phases. This section describes the results reported in a previous conference publication \cite{karulkarapplication}.