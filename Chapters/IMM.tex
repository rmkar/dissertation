\chapter{Model-based Approach to Estimate Gait Characteristics} \label{chapter:IMM}
The target population of this work is individuals with iSCIs, so the gait velocities of interest are very small in magnitude ($ \sim $ 0.5 m/s) \cite{nymark2005electromyographic}. There is limited work that uses physics-based models as a fundamental component in intent detection. The main attraction of using physics-based models is to allow for the accommodation of a greater number of users by modifying a small set of parameters such as mass, leg length, angles, and stiffness. This contribution uses the 3D B-SLIP model \cite{liu2015dynamic} as the basis of an estimation framework.

There are often physical changes to the gait as a result of changes in intended velocity, such as changes in step frequency, step length, and CoM trajectories \cite{kuo2001simple}. Template models emulate the salient characteristics of legged locomotion such as CoM trajectories and ground reaction force (GRF) profiles \cite{mochon1980ballistic}. Therefore, they may also emulate the changes in CoM trajectories and step length corresponding to changes in intent. The purpose of this contribution is to estimate gait characteristics such as gait phase and velocity through the comparison of sensor measurements to gaits from models of bipedal locomotion. The main idea behind the estimation framework developed is to estimate a person's desired gait velocity and phase by comparing the measured CoM trajectory to those in a library of B-SLIP gaits. The challenges were the increased difficulty in finding low-speed gaits during gait library generation and handling the hybrid dynamics of legged locomotion used to describe the different gait phases. This section describes the results reported in a previous conference publication \cite{karulkarapplication}.

\section{Generating the gait library}

Qualitative differences in the CoM trajectories of humans walking at different were observed in experimental data. Namely, the vertical and lateral excursion of the CoM over each step increase as gait velocity decreases. It was hypothesized that gait parameters may be estimated based on the changes in CoM trajectories. The 3D B-SLIP model presents a balance between model simplicity and CoM trajectory emulation. Additionally, using the 3D B-SLIP model allows the modeling of two important attributes of walking at low speeds; the lateral CoM movement that becomes increasingly dominant as gait velocity decreases and the DS period that represents a majority of the gait cycle. As the B-SLIP is a passive model, the omission of an actuator simplifies the estimation problem by reducing the assumptions necessary while applying these models to human walking. 

As described in Chapter~\ref{chapter:bg_info}, the state vector representing the CoM is $ \x_{CoM} = \left[x_{CoM} ,y_{CoM} ,z_{CoM} ,\dot{x}_{CoM} ,\dot{y}_{CoM} ,\dot{z}_{CoM}\right]^T $. The remaining parameters of the model, leg angles and stiffness, are collected in a parameter vector $ \greekvec{\zeta} = [\phi, \theta ,k ,z_{0_{CoM}} ,y_{0_{CoM}} ,\dot{x}_{0_{CoM}}]^T $ where $ \theta $ and $ \phi $ are leg angles that govern step length and width, and variables $ z_{0_{CoM}}$, $y_{0_{CoM}}$, and $\dot{x}_{0_{CoM}} $ give the vertical position, lateral position, and forward velocity of the CoM relative to the stance foot at the beginning of the step. The lack of actuators means optimization methods must be used to find periodic gaits for the B-SLIP model. An optimization procedure was carried out to find the parameters that would yield a fixed point for an MS-to-MS Poincar\'e map.

Gait optimization was considered via a nonlinear programming problem
%
\begin{eqnarray}
	\min_{\greekvec{\xi}}& \quad \lVert{\mat{J}(\greekvec{\xi})}\rVert ^2 + (k_{max} - k) \label{eq:gaitOpt}\\
	\textrm{s.t.}& \quad \vec{g}(\greekvec{\xi}) \leq \epsilon \nonumber
\end{eqnarray}
%
where $ \mat{J}(\greekvec{\xi}) $ was a vector-valued function returning the vertical excursion of the CoM, deviation from a desired step length and width, and the distance from the ground projection of the CoM to the foot at MS.
Nonlinear constraint functions $ \vec{g}(\vec{\xi}) $ ensured periodicity of the gait by matching the initial and final positions and velocities of the CoM with respect to the trailing foot to a tolerance $ \epsilon $. Constraints on the step length and width were found to be critical for the convergence of the optimization. Desired step length and width were approximated as a function of speed \cite{andriacchi1977walking} using a polynomial fit to human walking data \cite{fukuchi2018public}. As speed decreased, it was progressively difficult to choose appropriate parameters to seed the optimization, as the sensitivity of the model to initial conditions increased due to low passive stability at low speeds \cite{kuo2001simple}. This was an important challenge to overcome as gait speeds during rehabilitation are low \cite{seethapathi2015metabolic}. A predictor-corrector scheme was applied to start at a high-speed gait and compute appropriate seeds for low-speed gaits. 