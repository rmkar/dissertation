\chapter{Summary \& Conclusions}\label{chapter:conc}
\section{Summary}
Inferring user intent is an important step in achieving fluent HRI for robotic exoskeletons. Doing so predictively and intuitively will enable the robot to assist the user in realizing their desired changes while allowing for easy operation of the device. The objective of this dissertation was to develop frameworks to enable such user intent estimation for people with iSCIs using solely the sensors onboard the exoskeleton. Intent is an abstract notion so in this dissertation, the exoskeleton user's desired gait speed was assumed to be the expression of their intent. The framework presented in this dissertation exploits changes in gait patterns of exoskeleton users to infer the corresponding changes in their desired speed. As inter-subject gait variability and the severity of the subjects' injuries present challenges in estimating the user's desire to change speed, a data-driven strategy was presented to customize intent change estimation to fit each subject's individualized gait patterns.

Estimation of gait speed requires models of human locomotion along with models that correlate changes in gait features and speed. As such, this work draws on concepts from a variety of fields to use these models in appropriate estimation frameworks. The fundamentals of these concepts are described in Chapter~\ref{chapter:bg_info}. This chapter describes the Bipedal Spring Loaded Inverted Pendulum model of human locomotion, its hybrid dynamics, guard conditions, and reset maps. Poincar\'e surfaces and maps and their application to the B-SLIP model to analyze gait periodicity is also explained. Furthermore, as state estimation is a fundamental component of the objective of this dissertation, an explanation of discrete and extended Kalman Filters is also provided.

Accurate state estimation is necessary for fluent HRI and it can be performed using physics-based models. Human gait patterns can often be accurately described by reduced-order models such as the B-SLIP model. Chapter~\ref{chapter:IMM} describes a study of gait patterns observed during slow walking. A linearization-based gait search method is developed to handle the difficulty in optimizing low-speed gaits of the B-SLIP model, which is then used to generate a library of gaits that qualitatively match human gaits. This library is used in an estimation framework capable of handling the hybrid dynamics of the B-SLIP model. This framework compares sensor measurements from the exoskeleton with library gaits using a bank of parallel Extended Kalman Filters to estimate the most likely gait speed and gait phase. The insights gained from this work allow us to anticipate a user's desire to change their intended gait speed.

Foot placement while walking is highly informative of an individual's intended speed as step length varies with gait speed. A two-stage data-driven method to predict a user's desire to speed up or slow down in the future is described in Chapter~\ref{chapter:BKF}. This method leverages the relationship between user's gait speed and their interactions with the robot and the environment. Using this relationship, a prediction of user's desire to speed up or slow down is made at touchdown before the user realizes that change at the next midstance. This predictive nature of this method is in contrast to the reactive nature of many state-of-the-art user intent detection methods. User trials showed that the physical HRI is user-dependent which motivated personalization of this estimation framework. However, being data-driven the scarcity of of user-specific training data was a challenge in this approach. 

Data scarcity can be addressed by exploiting commonalities in gait patterns exist across users. Common patterns relating changes in gait speed to changes in gait features, e.g. changes in step length, are observed across exoskeleton users. Easily accessible walking data from healthy exoskeleton can be transformed and pooled with a small amount of novel, user-specific data to increase the available amount of training data while maintaining the user-specificity of the estimator. This pooling can be done with multiple different combinations of trial data and it is difficult to choose the appropriate combination manually. An approach that compares the mutual information between the gait speed and gait feature measurements to choose a combination of novel and base data as described in Chapter~\ref{chapter:MP}. This extended dataset was then used to create models correlating gait features to gait speed to be used in the estimation framework described in Chapter~\ref{chapter:BKF}. Data scarcity was addressed as the pooling process only used 8-12 steps' worth of novel user data.

\section{Future work}



\section{Conclusion}