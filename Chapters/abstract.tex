\begin{abstract}
	Collaboration between humans and robots has been increasing as society progresses. With this increased interaction Human-Robot Interaction (HRI), it is important for robots to understand the human's intent to assist them in completing their desired tasks. Enabling the human and the robots to accurately predict each others' actions is important to achieve a safe and successful collaboration. One class of HRI being studied is that of robotic exoskeletons used in the gait rehabilitation of people with incomplete Spinal Cord Injuries (iSCIs). As part of the rehabilitation process, an iSCI patient walks using a powered exoskeleton that assists them through their gait and allows them to relearn how to walk. The exoskeleton must be able to understand the user's intentions to achieve this objective. However, this process is hindered by several challenges such as the difficultly in quantifying intent as it is an abstract notion and gait variability arising due to the user's injuries.
	
	This dissertation aims to advance the state-of-the-art of user intent estimation for lower-limb exoskeletons. Existing strategies rely on rudimentary methods such as detecting weight transfer or torso tilt to initiate predefined gaits. The aim of this dissertation is to provide a more intuitive approach that uses additional gait variables to provide insight into the user's desired motion. 
	
	The primary mode of inferring user intent is through physical HRI (pHRI) as that is the most direct expression of the user's desired actions. To make the quantification of intent viable, the user's desired gait speed was assumed to be the indicator of their intent. It was shown that exoskeleton users' desire to change gait speed was apparent through their interactions with the robot and the environment. This insight enabled the development of a predictive two-stage estimation framework called the Buttressed Kalman Filter (BKF). This framework uses changes in the trends to gait features such as step length, frequency, and torso angles to infer the direction and magnitude of an exoskeleton user's desired future speed change. This predictive nature of the estimator is in contrast to many state-of-the-art estimators that are reactive and infer speed changes after they have happened.
	
	Gait patterns vary across individuals, making it difficult to generalize a single estimator across multiple users. This variability motivates the need to personalized estimators that are bespoke to an individuals gait patterns. This personalization of estimators was performed using a data-driven approach. The main challenge of this approach was scarcity of training data. This scarcity was addressed by transforming easily accessible user-independent base data from uninjured users with user-specific data from a novel user to increase the available training data. This pooling was made possible by exploiting commonalities in gait patterns that exist across individuals in spite of the gait variability. For example, changes in  step length and frequency show qualitatively similar trends relating to changes in desired velocity. This pooling method and BKF were tested on data collected during trials of injured users walking in an EksoGT exoskeleton. The average successful estimation of speed-up and slow-down changes improved from 52\% to 67\% when comparing with using base data alone by using only 8-12 steps' worth of data from the novel user with the best-case improvement of 32\%, from 48\% to 80\%.
	
\end{abstract}
