\begin{abstract}
	Collaboration between humans and robots has been increasing as society progresses. With this increased collaboration, intuitive Human-Robot Interaction (HRI) is important to enable humans and robots to accurately predict each others' actions to achieve safe and successful collaboration. One class of HRI being studied is that of robotic exoskeletons used in the gait rehabilitation of people with incomplete Spinal Cord Injuries (iSCIs). During that process, an iSCI patient walks using a powered exoskeleton that assists them through their gait and allows them to relearn how to walk. The exoskeleton must be able to understand the user's intent to achieve this objective. However, this process is hindered by several challenges such as the difficulty in quantifying intent, as it is an abstract notion, and gait variability arising due to the user's injuries. This dissertation aims to advance the state-of-the-art of user intent estimation for lower-limb exoskeletons. Existing strategies rely on rudimentary methods such as detecting weight transfer or torso tilt to initiate predefined gaits. This dissertation aims to provide a more intuitive approach that uses additional gait variables to provide insight into the user's desired motion. 
	
	To make the quantification of intent viable, the user's desired gait speed was assumed to be the indicator of their intent. This dissertation first describes a model-based method using Interacting Multi-Model (IMM) estimation to compare simulated steady-state gaits with measurements from the exoskeleton to infer the user's gait phase and speed. While the framework was able to estimate the gait phase and speed, this approach was reactive. Additionally, first principles cannot well model the decisions underlying and individual's realization of gait speed changes. 
	
	The primary mode of inferring user intent is through physical HRI (pHRI), as that is a direct expression of the user's desired actions. This dissertation further shows that exoskeleton users' desire to change gait speed was apparent through their interactions with the robot and the environment. Driven by this insight, the work herein describes a predictive two-stage estimation framework called the Buttressed Kalman Filter (BKF) that uses data-driven models to relate gait speed and gait features. This framework uses changes in the trends of gait features such as step length, frequency, and torso angles to infer the direction and magnitude of an exoskeleton user's desired future speed change. The predictive nature of the estimator is in contrast to many state-of-the-art estimators that are reactive and infer speed changes only after they have happened.
	
	Gait patterns vary across individuals, making it difficult to generalize a single estimator across multiple users and motivating the need for personalized estimators to suit an individual's gait patterns. However, the scarcity of individualized training data for people with iSCI presents a significant challenge. This dissertation details a method to address this scarcity by transforming easily accessible user-independent base data from uninjured users and pooling it with user-specific data from a novel user to create a larger dataset. %This pooling was possible due to commonalities in gait patterns that exist across individuals in spite of the gait variability. 
	The pooling method and BKF were tested on data collected during trials of injured users walking in an EksoGT exoskeleton. Using only 8-12 steps' worth of data from the novel user, the average successful estimation of speed-up and slow-down changes improved from 52\% to 67\% when compared to using base data alone with a best-case improvement of 32\%, from 48\% to 80\%. The contributions presented in this dissertation can potentially increase the ease of use of lower-limb exoskeletons for people with iSCIs.%
	\vspace{-5em}
\end{abstract}
